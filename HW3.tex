\documentclass{article}\usepackage[]{graphicx}\usepackage[]{xcolor}
% maxwidth is the original width if it is less than linewidth
% otherwise use linewidth (to make sure the graphics do not exceed the margin)
\makeatletter
\def\maxwidth{ %
  \ifdim\Gin@nat@width>\linewidth
    \linewidth
  \else
    \Gin@nat@width
  \fi
}
\makeatother

\definecolor{fgcolor}{rgb}{0.345, 0.345, 0.345}
\newcommand{\hlnum}[1]{\textcolor[rgb]{0.686,0.059,0.569}{#1}}%
\newcommand{\hlsng}[1]{\textcolor[rgb]{0.192,0.494,0.8}{#1}}%
\newcommand{\hlcom}[1]{\textcolor[rgb]{0.678,0.584,0.686}{\textit{#1}}}%
\newcommand{\hlopt}[1]{\textcolor[rgb]{0,0,0}{#1}}%
\newcommand{\hldef}[1]{\textcolor[rgb]{0.345,0.345,0.345}{#1}}%
\newcommand{\hlkwa}[1]{\textcolor[rgb]{0.161,0.373,0.58}{\textbf{#1}}}%
\newcommand{\hlkwb}[1]{\textcolor[rgb]{0.69,0.353,0.396}{#1}}%
\newcommand{\hlkwc}[1]{\textcolor[rgb]{0.333,0.667,0.333}{#1}}%
\newcommand{\hlkwd}[1]{\textcolor[rgb]{0.737,0.353,0.396}{\textbf{#1}}}%
\let\hlipl\hlkwb

\usepackage{framed}
\makeatletter
\newenvironment{kframe}{%
 \def\at@end@of@kframe{}%
 \ifinner\ifhmode%
  \def\at@end@of@kframe{\end{minipage}}%
  \begin{minipage}{\columnwidth}%
 \fi\fi%
 \def\FrameCommand##1{\hskip\@totalleftmargin \hskip-\fboxsep
 \colorbox{shadecolor}{##1}\hskip-\fboxsep
     % There is no \\@totalrightmargin, so:
     \hskip-\linewidth \hskip-\@totalleftmargin \hskip\columnwidth}%
 \MakeFramed {\advance\hsize-\width
   \@totalleftmargin\z@ \linewidth\hsize
   \@setminipage}}%
 {\par\unskip\endMakeFramed%
 \at@end@of@kframe}
\makeatother

\definecolor{shadecolor}{rgb}{.97, .97, .97}
\definecolor{messagecolor}{rgb}{0, 0, 0}
\definecolor{warningcolor}{rgb}{1, 0, 1}
\definecolor{errorcolor}{rgb}{1, 0, 0}
\newenvironment{knitrout}{}{} % an empty environment to be redefined in TeX

\usepackage{alltt}
\usepackage[margin=1.0in]{geometry} % To set margins
\usepackage{amsmath}  % This allows me to use the align functionality.
                      % If you find yourself trying to replicate
                      % something you found online, ensure you're
                      % loading the necessary packages!
\usepackage{amsfonts} % Math font
\usepackage{fancyvrb}
\usepackage{hyperref} % For including hyperlinks
\usepackage[shortlabels]{enumitem}% For enumerated lists with labels specified
                                  % We had to run tlmgr_install("enumitem") in R
\usepackage{float}    % For telling R where to put a table/figure
\usepackage{natbib}        %For the bibliography
\bibliographystyle{apalike}%For the bibliography
\IfFileExists{upquote.sty}{\usepackage{upquote}}{}
\begin{document}

\begin{enumerate}
%%%%%%%%%%%%%%%%%%%%%%%%%%%%%%%%%%%%%%%%%%%%%%%%%%%%%%%%%%%%%%%%%%%%%%%%%%%%%%%%
%%%%%%%%%%%%%%%%%%%%%%%%%%%%%%%%%%%%%%%%%%%%%%%%%%%%%%%%%%%%%%%%%%%%%%%%%%%%%%%%
% QUESTION 1
%%%%%%%%%%%%%%%%%%%%%%%%%%%%%%%%%%%%%%%%%%%%%%%%%%%%%%%%%%%%%%%%%%%%%%%%%%%%%%%%
%%%%%%%%%%%%%%%%%%%%%%%%%%%%%%%%%%%%%%%%%%%%%%%%%%%%%%%%%%%%%%%%%%%%%%%%%%%%%%%%
\item This week's Problem of the Week in Math is described as follows:
\begin{quotation}
  \textit{There are thirty positive integers less than 100 that share a certain 
  property. Your friend, Blake, wrote them down in the table to the left. But 
  Blake made a mistake! One of the numbers listed is wrong and should be replaced 
  with another. Which number is incorrect, what should it be replaced with, and 
  why?}
\end{quotation}
The numbers are listed below.
\begin{center}
  \begin{tabular}{ccccc}
    6 & 10 & 14 & 15 & 21\\
    22 & 26 & 33 & 34 & 35\\
    38 & 39 & 46 & 51 & 55\\
    57 & 58 & 62 & 65 & 69\\
    75 & 77 & 82 & 85 & 86\\
    87 & 91 & 93 & 94 & 95
  \end{tabular}
\end{center}
Use the fact that the ``certain'' property is that these numbers are all supposed
to be the product of \emph{unique} prime numbers to find and fix the mistake that
Blake made.\\
\textbf{Reminder:} Code your solution in an \texttt{R} script and copy it over
to this \texttt{.Rnw} file.\\
\textbf{Hint:} You may find the \verb|%in%| operator and the \verb|setdiff()| function to be helpful.\\

\textbf{Solution:} 
% Write your answer and explanations here.

\begin{knitrout}\scriptsize
\definecolor{shadecolor}{rgb}{0.969, 0.969, 0.969}\color{fgcolor}\begin{kframe}
\begin{alltt}
\hldef{original.numbers} \hlkwb{<-} \hlkwd{c}\hldef{(}\hlnum{6}\hldef{,}\hlnum{10}\hldef{,}\hlnum{14}\hldef{,}\hlnum{15}\hldef{,}\hlnum{21}\hldef{,}\hlnum{22}\hldef{,}\hlnum{26}\hldef{,}\hlnum{33}\hldef{,}\hlnum{34}\hldef{,}\hlnum{35}\hldef{,}
                  \hlnum{38}\hldef{,}\hlnum{39}\hldef{,}\hlnum{46}\hldef{,}\hlnum{51}\hldef{,}\hlnum{55}\hldef{,}\hlnum{57}\hldef{,}\hlnum{58}\hldef{,}\hlnum{62}\hldef{,}\hlnum{65}\hldef{,}\hlnum{69}\hldef{,}
                  \hlnum{75}\hldef{,}\hlnum{77}\hldef{,}\hlnum{82}\hldef{,}\hlnum{85}\hldef{,}\hlnum{86}\hldef{,}\hlnum{87}\hldef{,}\hlnum{91}\hldef{,}\hlnum{93}\hldef{,}\hlnum{94}\hldef{,}\hlnum{95}\hldef{)}

\hldef{prime.numbers} \hlkwb{<-} \hlkwd{c}\hldef{(}\hlnum{2}\hldef{,}\hlnum{3}\hldef{,}\hlnum{5}\hldef{,}\hlnum{7}\hldef{,}\hlnum{11}\hldef{,}\hlnum{13}\hldef{,}\hlnum{17}\hldef{,}\hlnum{19}\hldef{,}\hlnum{23}\hldef{,}\hlnum{29}\hldef{,}\hlnum{31}\hldef{,}
                   \hlnum{37}\hldef{,}\hlnum{41}\hldef{,}\hlnum{43}\hldef{,}\hlnum{47}\hldef{,}\hlnum{53}\hldef{)}

\hldef{products} \hlkwb{<-} \hlkwd{c}\hldef{()}
\hlkwa{for}\hldef{(i} \hlkwa{in} \hlnum{1}\hlopt{:}\hldef{(}\hlkwd{length}\hldef{(prime.numbers)}\hlopt{-}\hlnum{1}\hldef{))\{}
  \hldef{curr} \hlkwb{<-} \hldef{prime.numbers[i]}
  \hlcom{# print(prime.numbers[(i+1):length(prime.numbers)] *curr)}
  \hlkwa{for}\hldef{(j} \hlkwa{in} \hldef{prime.numbers[(i}\hlopt{+}\hlnum{1}\hldef{)}\hlopt{:}\hlkwd{length}\hldef{(prime.numbers)])\{}
    \hldef{temp} \hlkwb{<-} \hldef{j}\hlopt{*}\hldef{curr}
    \hlkwa{if}\hldef{(temp} \hlopt{<} \hlnum{100}\hldef{)\{}
      \hldef{products} \hlkwb{<-} \hlkwd{append}\hldef{(products,temp)}
    \hldef{\}}
    \hlkwa{else}\hldef{\{}
      \hlkwa{break}
    \hldef{\}}
  \hldef{\}}
\hldef{\}}
\hldef{wrong.answer} \hlkwb{<-} \hlkwd{setdiff}\hldef{(original.numbers,products)}
\hldef{correct.replacement} \hlkwb{<-} \hlkwd{setdiff}\hldef{(products,original.numbers)}
\end{alltt}
\end{kframe}
\end{knitrout}
\end{enumerate}

\bibliography{bibliography}
\end{document}
